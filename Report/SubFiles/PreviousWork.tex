\section{Previous Work}
The area of text categorization is highly studied and hence many works have been presented, however an optimal feature set has not yet been published \cite{1}. This may be due to different domains having different features or due to the complexity of the problem. \cite{1} uses RNN models for author identification. A particular emphasis is made on GloVe vectors. This method extracts both context, since it uses the actual words, as well as style, due to the RNN storing the sequence and hence the particular order in which the words are used. For this work however, context will not be taken into account, hence there is no use of the raw words themselves.

In \cite{3}, the difference between the style of writing of the two different genders in the domain of fiction and non-fiction works has been studied. It was noted that the main distinguishing features are different between the two domains. On the other hand, it was discovered that certain function words as well as parts-of-speech(POS) tags can be used to discriminate between the two genders in both domains. In this work, both of these feature sets are used applied to the blogging domain.

\cite{4} applies the task of gender identification within the e-mail domain. Here, the language background of the author is also considered and classified. It is stated that females tend to use more expressive language. Some of the words used mostly be females include those such as "terribly" and "dreadful". In this work, these are accounted for by taking into account words ending with certain suffixes. A general distinction between the styles of men and females within the e-mail domain is that male conversation tends to resemble 'report-talk' while females prefer 'rapport-talk'.

\cite{2} use the same corpus as this work and define the state of the art of this task. They claim that women write in a style which is more involved, meanwhile, men prefer writing which gives more information, which is similar to the 'report' and 'rapport' talk discussed by \cite{4}. Important features such as the number of URL links and use of pronouns are discussed, and some of these will be used within this work. \cite{2} also uses the topics and individual words of the posts. This is something which this work will avoid, in order to study how well the gender can be predicted using purely style-based features without the bias of topic. The observation that style changes as the age changes is also made. This means that as age differs, the style between the gender also varies and mix into each other, making it more difficult to predict gender based purely off of stylistic features.