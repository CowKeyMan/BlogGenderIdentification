\documentclass{article}
\usepackage{graphicx}
\usepackage{listings}
\usepackage{color}
\usepackage[margin=1.2in]{geometry}
\usepackage{courier}
\usepackage{float}
\usepackage[toc,page]{appendix}
\usepackage{amsmath}
\usepackage{hyperref}
\usepackage{booktabs}
\usepackage{multirow}
\usepackage{setspace}
\doublespacing
\large

\definecolor{codegreen}{rgb}{0,0.6,0}
\definecolor{codeblack}{rgb}{0,0,0}
\definecolor{codered}{rgb}{0.867,0,0}
\definecolor{backcolour}{rgb}{ 0.95,0.95,0.95}
\definecolor{codeorange}{rgb}{1,0.447059,0}
\definecolor{codegrey}{rgb}{0.1, 0.1, 0.1}
\definecolor{codeblue}{rgb}{0.29, 0.52, 0.56}
\lstdefinestyle{mystyle}{
    basicstyle=\ttfamily\color{codeblack},
    backgroundcolor=\color{backcolour},   
    commentstyle=\color{codered},
    keywordstyle=\color{codeorange},
    numberstyle=\color{codeblack},
    stringstyle=\color{codegreen},
    breakatwhitespace=false,         
    breaklines=true,                 
    captionpos=b,                    
    keepspaces=true,                 
    numbers=left,                    
    numbersep=10pt,                  
    showspaces=false,                
    showstringspaces=false,
    showtabs=false,                  
    tabsize=3
}
 
\lstset{style=mystyle}

\begin{document}
\title{A Style-Based Approach to Gender Identification on Blogs}
\author{Daniel Cauchi}
\date{}
\maketitle

\section*{Abstract}
While having bias is necessary to generalising beyond training data, it may sometimes lead to overfitting. Within the context of gender identification in text, more specifically in the blogging domain, one of these biases is usually the topic of the individual documents. This work attempts to study the problem of gender classification in text by using solely stylistic features.

\section{Introduction}
In this study, the task of Gender Identification on a Blog Corpus is implemented trough the use of Natural Language Processing techniques. The goal is to build a classifier which is able to recognize the differences in writing style between male and female writers. Usually, in other natural language tasks such as text generation, the issue of gender bias is one which researchers try to avoid, in order to prevent the model from learning stereotypes. In this case, however, the opposite is required, where the features will be chosen to amplify this bias. The task is to find the ideal features which will allow the generated model to discriminate best between male and female writers. For this approach, any feature which is related to topic is avoided, purely focusing on the style of writing. This is particularly difficult because the aim is to try and find similarities across an entire group of authors, because while individuals may have their own style of writing, it is more difficult to generalise the style over a set of people \cite{3}.

The solution to the gender identification problem has applications in areas such as forensics \cite{2} and cybercriminal analysis \cite{1}, so the identity of a suspect may be narrowed down to one gender. Another application is in marketing, to find the gender demographic of some outlet.

The following chapters will discuss previous work done in this area, a description of the corpus in question, the approach taken to solve this problem and an evaluation on the model built.

\section{Previous Work}
The area of text categorization is highly studied and hence many works have been presented, however an optimal feature set hs not yet been published \cite{1}. This may be due to different domains having different features or due to the complexity of the problem. \cite{1} Uses RNN models for author identification. A particular emphasis is made on GloVe vectors. This method extracts both context, since it uses the actual words, as well as style, due to the RNN storing the sequence and hence the particular order in which the words are used. For this work however, context will not be taken into account, hence there is no use of topic specific words.

In \cite{3}, the authors of fictional and non-fictional works has been studied. It was noted that the main distinguishing features are different between the two domains. It was discovered that certain function words as well as parts-of-speech can be used to discriminate between the two genders in both domains. In this project, both of these feature sets are used applied to the blogging domain.

\cite{4} applies the task of gender identification for the e-mail domain. In this work, the language background of the author is also classified. It is stated that females tend to use more expressive language. These words include those such as "terribly" and "dreadful". In this project, these are accounted for by taking into account words ending with certain suffixes. A general distinction between the styles of men and females is that while males' conversation tend to prefer 'report-talk' while females prefer 'rapport-talk'.

\cite{2} use the same corpus as this project and define the state of the art of this task. They claim that women write in a style which is more involved, meanwhile men prefer writing which give more information, which is similar to the 'report' and 'rapport' talk discussed by \cite{4}. Important features such as the number of URL links and use of pronouns are discussed, and some of these will be used within this project \cite{2} also uses the topics and individual words of the posts. This is something which this work will avoid, to try and see how well the gender can be predicted using purely style based features without the bias of topic. The observation that style changes as the age changes is also made. This means that as age differs, the style between the gender also varies and blend into each other, making it more difficult to extract gender based purely off of stylistic features.
\section{Data and Methodology}

The corpus used is the Blog Authorship Corpus by \cite{2}. This is a corpus containing posts from 19320 blogs, with 9660 of them being male and another 9660 of them being female. In total there are 344773 male posts and 335010 posts. Other meta-data includes the age range of the bloggers and their star sign. These will not be considered. Other observations on the data include that there are some null posts, which will be pruned later on. Furthermore, some blogs are pure spam, with posts containing garbage such as advertisements or non-sense sentences. These will act as noise in the data. 

The following 'general' features were extracted from the corpus for each post: Word Count, Sentence Count, Average Word Length, Average Sentence Length, Number of Unique Words/Word Count, Number of URLs/Word Count. The number of words ending with the following suffixes were also counted and divided by the Word Count: able, al, ful, ible, ic, ive, less, ly, ous. This list was obtained from \cite{4} The reason they were divided by the Word Count is because the length of a post may affect the frequency of words in general, hence the features are normalize by this value.

A few vocabularies were created to extract some important words and bi-grams with regards to the posts. Prior to this however, the posts were split into a train and test set.

% Discuss how data was bleached and pos were obtained, how bigrams were used. only done on training set, and those with high frequency but lowest entropy (highest IG) were chosen. How training and testing set were split.
\section{Results and Evaluation}
The confusion matrix of the baseline model can be seen in table \ref{tbl:baseline} while that of the main model can be seen in table \ref{tbl:main}. The accuracy of the baseline ended up being 60\% while that of the main model ended up being the same at 60\%. These results are very similar and based on the fact that the overall accuracy is above 50\% on a significant amount of data shows that some form of pattern was found. However, this score is still very inconclusive, especially when compared to the state-of-the-art presented in \cite{2} which achieved an overall accuracy of 80.1\%. Scaling was also later done on the data using the Standard Scaler provided by scikit-learn using the settings 'with mean' and 'with std', but the results were very similar. Next, the reasons as to why these results may have occurred are discussed.

\begin{table}[] \label{tbl:baseline}
	\renewcommand{\arraystretch}{1.3}
	\centering
	\begin{tabular}{|c|l|l|l|}
		\hline
		\multirow{3}{*}{\textbf{Actual}} & \textbf{Male}   & 36731 (57\%)  & 28067 (43\%)    \\ \cline{2-4} 
		& \textbf{Female} & 27108 (36\%)  & 47192 (64\%)    \\ \cline{2-4} 
		&                 & \textbf{Male} & \textbf{Female} \\ \hline
		\multicolumn{1}{|l|}{}           & \multicolumn{3}{c|}{\textbf{Predicted}}           \\ \hline
	\end{tabular}
	\caption{Logistic Regression Model Confusion Matrix}
\end{table}

\begin{table}[]\label{tbl:main}
	\renewcommand{\arraystretch}{1.3}
	\centering
	\begin{tabular}{|c|l|l|l|}
		\hline
		\multirow{3}{*}{\textbf{Actual}} & \textbf{Male}   & 37067 (57\%)  & 27731 (43\%)    \\ \cline{2-4} 
		& \textbf{Female} & 28352 (38\%)  & 45948 (62\%)    \\ \cline{2-4} 
		&                 & \textbf{Male} & \textbf{Female} \\ \hline
		\multicolumn{1}{|l|}{}           & \multicolumn{3}{c|}{\textbf{Predicted}}           \\ \hline
	\end{tabular}
	\caption{Multi Layer Perceptron Model Confusion Matrix}
\end{table}

Firstly, the reason why the state-of-the-art is so low is considered. This may be due to the fact that style varies by age, as pointed out by \cite{2}. It is stated that  as people grow older, the style starts resembling that of the male writing more and more. Therefore, as the data contains writers of mixed ages, it becomes increasingly difficult to distinguish between male and female writers. Another obstacle is the fact that each age group does not have the same amount of writers from both genders. In fact, there are more female teenage bloggers, while the older age range consists of mostly men. The final hurdle is the noise in the data, that is, the blogs with spam for content, as well as those bloggers whose authors have provided false information with regards to their identification \cite{2}.

Next, the reason why this implementation yielded worse results is discussed. As pointed out by \cite{2}, using both style and content-based features yielded the best results. Content-based features however were deliberately avoided, to account for men and women writing about the same things. Furthermore, the number of bi-gram and function word features was trimmed down, so using more of these might increase accuracy. This, however, is likely to provide nothing more than a slight improvement, due to the fact that both frequency and information gain were taken into account, so the remaining bi-grams and function words should have little effect and also lead to more sparseness in the data. 
\section{Future Work}
While this study has revealed some potentially useful features, more work may be done to build upon it. One way is to repeat the process but prune less features. Another is to approach the feature selection for the POS and bleached features differently, such as using more than one n-gram type and uncovering distinguishing features in other n-gram types.

Other potentially useful features are the use of misspelt words, the use of excessive punctuation or the use of emojis. Other forms of bleaching presented in \cite{5} may also be tested, with different n-gram types. Other studies may also opt to perform the work using only a subset of the data, such as considering only one age range. % different n-grams, using more features, testing with specific ge ranges only, filtering the data for spam accounts

\section{Conclusion}
This study  has presented a way of distinguishing between gender writing styles in the blogging domain. The method presented does not consider topic, but rather only makes use of style-based features. As such, it does not improve upon the state-of-the-art, but presents a new way of trying to tackle this problem. Almost all the features used, except for a few such as 'Number of URL links' could have been used in other text domains, besides blogging. Although the decisions made were done deliberately to try and reduce the amount of bias of the domain as much as possible, the classification score suffers as a result.

While much has been studied already, it is clear that more work needs to be done in this area to determine which features may distinguish well between the style of writing between the genders. This is especially the case for the less biased approach of classification, where style is the only focus, without the effect of topic.

\pagebreak

\bibliography{main}
\bibliographystyle{ieeetr}

\pagebreak

\chapter*{Appendix}\addcontentsline{toc}{section}{Appendix}

\section*{Plagiarism Form}
\begin{figure}[H]
	\centering
	\includegraphics[width=0.7\textwidth]{Images/plagiarismform.png}
\end{figure}

\pagebreak

\section*{Completion Form}
\begin{table}[h]
	\begin{tabular}{|l|l|}
		\hline
		\textbf{Item}                                     & \textbf{Completed} (Yes/No/Partial) \\ \hline
		&                            \\ \hline
		SVM technical discussion     & Yes                        \\ \hline
		A good comparison to alternative methods                  & Yes                        \\ \hline
		Artifact                  & Yes                        \\ \hline
		Experimentation with different SVM parameters                  & Yes                        \\ \hline
		Experiments and their evaluation & Yes                        \\ \hline
		Overall conclusions           & Yes                        \\ \hline
	\end{tabular}
\end{table}

\end{document}