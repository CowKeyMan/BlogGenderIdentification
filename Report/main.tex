\documentclass{article}
\usepackage{graphicx}
\usepackage{listings}
\usepackage{color}
\usepackage[margin=1.2in]{geometry}
\usepackage{courier}
\usepackage{float}
\usepackage[toc,page]{appendix}
\usepackage{amsmath}
\usepackage{hyperref}
\usepackage{setspace}
\doublespacing
\large

\definecolor{codegreen}{rgb}{0,0.6,0}
\definecolor{codeblack}{rgb}{0,0,0}
\definecolor{codered}{rgb}{0.867,0,0}
\definecolor{backcolour}{rgb}{ 0.95,0.95,0.95}
\definecolor{codeorange}{rgb}{1,0.447059,0}
\definecolor{codegrey}{rgb}{0.1, 0.1, 0.1}
\definecolor{codeblue}{rgb}{0.29, 0.52, 0.56}
\lstdefinestyle{mystyle}{
    basicstyle=\ttfamily\color{codeblack},
    backgroundcolor=\color{backcolour},   
    commentstyle=\color{codered},
    keywordstyle=\color{codeorange},
    numberstyle=\color{codeblack},
    stringstyle=\color{codegreen},
    breakatwhitespace=false,         
    breaklines=true,                 
    captionpos=b,                    
    keepspaces=true,                 
    numbers=left,                    
    numbersep=10pt,                  
    showspaces=false,                
    showstringspaces=false,
    showtabs=false,                  
    tabsize=3
}
 
\lstset{style=mystyle}

\begin{document}
\title{LIN3012 - Data-Driven Natural Language Processing}
\author{Daniel Cauchi}
\date{}
\maketitle

%\tableofcontents

\section{Introduction}
In this report, the task of Gender Identification on a Blog Corpus is implemented trough the use of Natural Language Processing techniques. The goal is to build a classifier which is able to recognize the differences in writing style between male and female writers. The problem is to find the ideal features which will allow the classifier to discriminate best between male and female writers. For this approach, any feature which is related to topic is avoided, purely focusing on the style of writing. This is particularly difficult because the aim is to try and find similarities across an entire group of authors, because while individuals may have their own style of writing, it is more difficult to generalise the style of a set of people \cite{3}.


The solution to the gender identification problem has applications in areas such as forensics \cite{2, 6} and cybercriminal analysis \cite{1}, so the idenity of a suspect may be narrowed down to one gender. Another application is in marketing \cite{6}, to find the demographic of some outlet.

The following chapters will discuss previous work done in this area, a description of the corpus in question, the approach taken to solve this problem and an evaluation on the model built.

\section{Previous Work}
The area of text categorization is highly studied and hence many works have been presented, however an optimal feature set hs not yet been published \cite{1}. This may be due to different domains having different features or due to the complexity of the problem. \cite{1} Uses RNN models for author identification. A particular emphasis is made on GloVe vectors. This method extracts both context, since it uses the actual words, as well as style, due to the RNN storing the sequence and hence the particular order in which the words are used. For this work however, context will not be taken into account, hence there is no use of topic specific words.

In \cite{3}, the authors of fictional and non-fictional works has been studied. It was noted that the main distinguishing features are different between the two domains. It was discovered that certain function words as well as parts-of-speech can be used to discriminate between the two genders in both domains. In this project, both of these feature sets are used applied to the blogging domain.

\cite{4} applies the task of gender identification for the e-mail domain. In this work, the language background of the author is also classified. It is stated that females tend to use more expressive language. These words include those such as "terribly" and "dreadful". In this project, these are accounted for by taking into account words ending with certain suffixes. A general distinction between the styles of men and females is that while males' conversation tend to prefer 'report-talk' while females prefer 'rapport-talk'.

\cite{2} use the same corpus as this project and define the state of the art of this task. They claim that women write in a style which is more involved, meanwhile men prefer writing which give more information, which is similar to the 'report' and 'rapport' talk discussed by \cite{4}. Important features such as the number of URL links and use of pronouns are discussed, and some of these will be used within this project \cite{2} also uses the topics and individual words of the posts. This is something which this work will avoid, to try and see how well the gender can be predicted using purely style based features without the bias of topic. The observation that style changes as the age changes is also made. This means that as age differs, the style between the gender also varies and blend into each other, making it more difficult to extract gender based purely off of stylistic features.

\chapter{Conclusion}


\bibliography{main}
\bibliographystyle{ieeetr}

%\chapter*{Appendix}\addcontentsline{toc}{section}{Appendix}

\section*{Plagiarism Form}
\begin{figure}[H]
	\centering
	\includegraphics[width=0.7\textwidth]{Images/plagiarismform.png}
\end{figure}

\pagebreak

\section*{Completion Form}
\begin{table}[h]
	\begin{tabular}{|l|l|}
		\hline
		\textbf{Item}                                     & \textbf{Completed} (Yes/No/Partial) \\ \hline
		&                            \\ \hline
		SVM technical discussion     & Yes                        \\ \hline
		A good comparison to alternative methods                  & Yes                        \\ \hline
		Artifact                  & Yes                        \\ \hline
		Experimentation with different SVM parameters                  & Yes                        \\ \hline
		Experiments and their evaluation & Yes                        \\ \hline
		Overall conclusions           & Yes                        \\ \hline
	\end{tabular}
\end{table}

\end{document}